%% Generated by Sphinx.
\def\sphinxdocclass{report}
\documentclass[letterpaper,10pt,english]{sphinxmanual}
\ifdefined\pdfpxdimen
   \let\sphinxpxdimen\pdfpxdimen\else\newdimen\sphinxpxdimen
\fi \sphinxpxdimen=.75bp\relax

\PassOptionsToPackage{warn}{textcomp}
\usepackage[utf8]{inputenc}
\ifdefined\DeclareUnicodeCharacter
% support both utf8 and utf8x syntaxes
  \ifdefined\DeclareUnicodeCharacterAsOptional
    \def\sphinxDUC#1{\DeclareUnicodeCharacter{"#1}}
  \else
    \let\sphinxDUC\DeclareUnicodeCharacter
  \fi
  \sphinxDUC{00A0}{\nobreakspace}
  \sphinxDUC{2500}{\sphinxunichar{2500}}
  \sphinxDUC{2502}{\sphinxunichar{2502}}
  \sphinxDUC{2514}{\sphinxunichar{2514}}
  \sphinxDUC{251C}{\sphinxunichar{251C}}
  \sphinxDUC{2572}{\textbackslash}
\fi
\usepackage{cmap}
\usepackage[T1]{fontenc}
\usepackage{amsmath,amssymb,amstext}
\usepackage{babel}



\usepackage{times}
\expandafter\ifx\csname T@LGR\endcsname\relax
\else
% LGR was declared as font encoding
  \substitutefont{LGR}{\rmdefault}{cmr}
  \substitutefont{LGR}{\sfdefault}{cmss}
  \substitutefont{LGR}{\ttdefault}{cmtt}
\fi
\expandafter\ifx\csname T@X2\endcsname\relax
  \expandafter\ifx\csname T@T2A\endcsname\relax
  \else
  % T2A was declared as font encoding
    \substitutefont{T2A}{\rmdefault}{cmr}
    \substitutefont{T2A}{\sfdefault}{cmss}
    \substitutefont{T2A}{\ttdefault}{cmtt}
  \fi
\else
% X2 was declared as font encoding
  \substitutefont{X2}{\rmdefault}{cmr}
  \substitutefont{X2}{\sfdefault}{cmss}
  \substitutefont{X2}{\ttdefault}{cmtt}
\fi


\usepackage[Bjarne]{fncychap}
\usepackage{sphinx}

\fvset{fontsize=\small}
\usepackage{geometry}


% Include hyperref last.
\usepackage{hyperref}
% Fix anchor placement for figures with captions.
\usepackage{hypcap}% it must be loaded after hyperref.
% Set up styles of URL: it should be placed after hyperref.
\urlstyle{same}


\usepackage{sphinxmessages}




\title{octapy}
\date{Apr 27, 2021}
\release{1.0}
\author{Jason Tilley}
\newcommand{\sphinxlogo}{\vbox{}}
\renewcommand{\releasename}{Release}
\makeindex
\begin{document}

\pagestyle{empty}
\sphinxmaketitle
\pagestyle{plain}
\sphinxtableofcontents
\pagestyle{normal}
\phantomsection\label{\detokenize{octapy::doc}}



\chapter{Submodules}
\label{\detokenize{octapy:submodules}}

\chapter{octapy.data module}
\label{\detokenize{octapy:module-octapy.data}}\label{\detokenize{octapy:octapy-data-module}}\index{module@\spxentry{module}!octapy.data@\spxentry{octapy.data}}\index{octapy.data@\spxentry{octapy.data}!module@\spxentry{module}}\index{Data (class in octapy.data)@\spxentry{Data}\spxextra{class in octapy.data}}

\begin{fulllineitems}
\phantomsection\label{\detokenize{octapy:octapy.data.Data}}\pysigline{\sphinxbfcode{\sphinxupquote{class }}\sphinxcode{\sphinxupquote{octapy.data.}}\sphinxbfcode{\sphinxupquote{Data}}}
\sphinxAtStartPar
Bases: \sphinxcode{\sphinxupquote{object}}

\sphinxAtStartPar
a Cython data container class
\index{sal (octapy.data.Data attribute)@\spxentry{sal}\spxextra{octapy.data.Data attribute}}

\begin{fulllineitems}
\phantomsection\label{\detokenize{octapy:octapy.data.Data.sal}}\pysigline{\sphinxbfcode{\sphinxupquote{sal}}}
\sphinxAtStartPar
salinity

\end{fulllineitems}

\index{ssh (octapy.data.Data attribute)@\spxentry{ssh}\spxextra{octapy.data.Data attribute}}

\begin{fulllineitems}
\phantomsection\label{\detokenize{octapy:octapy.data.Data.ssh}}\pysigline{\sphinxbfcode{\sphinxupquote{ssh}}}
\sphinxAtStartPar
sea surface height

\end{fulllineitems}

\index{temp (octapy.data.Data attribute)@\spxentry{temp}\spxextra{octapy.data.Data attribute}}

\begin{fulllineitems}
\phantomsection\label{\detokenize{octapy:octapy.data.Data.temp}}\pysigline{\sphinxbfcode{\sphinxupquote{temp}}}
\sphinxAtStartPar
temperature

\end{fulllineitems}

\index{u (octapy.data.Data attribute)@\spxentry{u}\spxextra{octapy.data.Data attribute}}

\begin{fulllineitems}
\phantomsection\label{\detokenize{octapy:octapy.data.Data.u}}\pysigline{\sphinxbfcode{\sphinxupquote{u}}}
\sphinxAtStartPar
u\sphinxhyphen{}velocity (eastward velocity)

\end{fulllineitems}

\index{v (octapy.data.Data attribute)@\spxentry{v}\spxextra{octapy.data.Data attribute}}

\begin{fulllineitems}
\phantomsection\label{\detokenize{octapy:octapy.data.Data.v}}\pysigline{\sphinxbfcode{\sphinxupquote{v}}}
\sphinxAtStartPar
v\sphinxhyphen{}velocity (northward velocity)

\end{fulllineitems}

\index{w (octapy.data.Data attribute)@\spxentry{w}\spxextra{octapy.data.Data attribute}}

\begin{fulllineitems}
\phantomsection\label{\detokenize{octapy:octapy.data.Data.w}}\pysigline{\sphinxbfcode{\sphinxupquote{w}}}
\sphinxAtStartPar
w\sphinxhyphen{}velocity (upward velocity)

\end{fulllineitems}


\end{fulllineitems}



\chapter{octapy.get\_data\_at\_index module}
\label{\detokenize{octapy:module-octapy.get_data_at_index}}\label{\detokenize{octapy:octapy-get-data-at-index-module}}\index{module@\spxentry{module}!octapy.get\_data\_at\_index@\spxentry{octapy.get\_data\_at\_index}}\index{octapy.get\_data\_at\_index@\spxentry{octapy.get\_data\_at\_index}!module@\spxentry{module}}\index{get\_data\_at\_index() (in module octapy.get\_data\_at\_index)@\spxentry{get\_data\_at\_index()}\spxextra{in module octapy.get\_data\_at\_index}}

\begin{fulllineitems}
\phantomsection\label{\detokenize{octapy:octapy.get_data_at_index.get_data_at_index}}\pysiglinewithargsret{\sphinxcode{\sphinxupquote{octapy.get\_data\_at\_index.}}\sphinxbfcode{\sphinxupquote{get\_data\_at\_index}}}{}{}
\sphinxAtStartPar
Cython function for getting  data from a netCDF file at a given index
location
\begin{quote}\begin{description}
\item[{Parameters}] \leavevmode\begin{itemize}
\item {} 
\sphinxAtStartPar
\sphinxstyleliteralstrong{\sphinxupquote{filepath}} \textendash{} path to the netCDF file

\item {} 
\sphinxAtStartPar
\sphinxstyleliteralstrong{\sphinxupquote{index}} \textendash{} the ravelled index to extract 3D variables in the netCDF file

\item {} 
\sphinxAtStartPar
\sphinxstyleliteralstrong{\sphinxupquote{surf\_index}} \textendash{} the ravelled index to extract sea surface height in the netCDF file

\item {} 
\sphinxAtStartPar
\sphinxstyleliteralstrong{\sphinxupquote{dims}} \textendash{} the dimensions of the model

\end{itemize}

\item[{Returns}] \leavevmode
\sphinxAtStartPar
a Data object instance

\end{description}\end{quote}

\end{fulllineitems}



\chapter{octapy.interp\_idw module}
\label{\detokenize{octapy:module-octapy.interp_idw}}\label{\detokenize{octapy:octapy-interp-idw-module}}\index{module@\spxentry{module}!octapy.interp\_idw@\spxentry{octapy.interp\_idw}}\index{octapy.interp\_idw@\spxentry{octapy.interp\_idw}!module@\spxentry{module}}\index{interp\_idw() (in module octapy.interp\_idw)@\spxentry{interp\_idw()}\spxextra{in module octapy.interp\_idw}}

\begin{fulllineitems}
\phantomsection\label{\detokenize{octapy:octapy.interp_idw.interp_idw}}\pysiglinewithargsret{\sphinxcode{\sphinxupquote{octapy.interp\_idw.}}\sphinxbfcode{\sphinxupquote{interp\_idw}}}{}{}
\sphinxAtStartPar
a Cython inverse distance weighted interpolation function
\begin{quote}\begin{description}
\item[{Parameters}] \leavevmode\begin{itemize}
\item {} 
\sphinxAtStartPar
\sphinxstyleliteralstrong{\sphinxupquote{particle}} \textendash{} a Particle instance

\item {} 
\sphinxAtStartPar
\sphinxstyleliteralstrong{\sphinxupquote{data}} \textendash{} a Data instance

\item {} 
\sphinxAtStartPar
\sphinxstyleliteralstrong{\sphinxupquote{weights}} \textendash{} weights used for inverse distance weighted interpolation

\item {} 
\sphinxAtStartPar
\sphinxstyleliteralstrong{\sphinxupquote{dims}} \textendash{} dimensions of the Model instance

\end{itemize}

\item[{Returns}] \leavevmode
\sphinxAtStartPar
a particle instance with interpolated environmental data

\end{description}\end{quote}

\end{fulllineitems}



\chapter{octapy.tools module}
\label{\detokenize{octapy:module-octapy.tools}}\label{\detokenize{octapy:octapy-tools-module}}\index{module@\spxentry{module}!octapy.tools@\spxentry{octapy.tools}}\index{octapy.tools@\spxentry{octapy.tools}!module@\spxentry{module}}\index{get\_filepath() (in module octapy.tools)@\spxentry{get\_filepath()}\spxextra{in module octapy.tools}}

\begin{fulllineitems}
\phantomsection\label{\detokenize{octapy:octapy.tools.get_filepath}}\pysiglinewithargsret{\sphinxcode{\sphinxupquote{octapy.tools.}}\sphinxbfcode{\sphinxupquote{get\_filepath}}}{\emph{\DUrole{n}{datetime64}}, \emph{\DUrole{n}{model\_name}}, \emph{\DUrole{n}{submodel\_name}}, \emph{\DUrole{n}{data\_dir}}}{}
\sphinxAtStartPar
Get the filename for a given timestep
\begin{quote}\begin{description}
\item[{Parameters}] \leavevmode\begin{itemize}
\item {} 
\sphinxAtStartPar
\sphinxstyleliteralstrong{\sphinxupquote{datetime64}} \textendash{} a numpy.datetime64 object for the data’s time

\item {} 
\sphinxAtStartPar
\sphinxstyleliteralstrong{\sphinxupquote{model\_name}} \textendash{} the oceanographic model being used

\end{itemize}

\item[{Returns}] \leavevmode
\sphinxAtStartPar
string of the filepath

\end{description}\end{quote}

\end{fulllineitems}

\index{get\_extent() (in module octapy.tools)@\spxentry{get\_extent()}\spxextra{in module octapy.tools}}

\begin{fulllineitems}
\phantomsection\label{\detokenize{octapy:octapy.tools.get_extent}}\pysiglinewithargsret{\sphinxcode{\sphinxupquote{octapy.tools.}}\sphinxbfcode{\sphinxupquote{get\_extent}}}{\emph{\DUrole{n}{grid}}}{}
\sphinxAtStartPar
Get the spatial extent of a grid as a list
\begin{quote}\begin{description}
\item[{Parameters}] \leavevmode
\sphinxAtStartPar
\sphinxstyleliteralstrong{\sphinxupquote{grid}} \textendash{} a Grid instance

\item[{Returns}] \leavevmode
\sphinxAtStartPar
a list of coordinates representing the extent as {[}minimum longitude,
maximum longitude, minimum latitude, maximum latitude{]}

\end{description}\end{quote}

\end{fulllineitems}

\index{netcdf\_to\_csv() (in module octapy.tools)@\spxentry{netcdf\_to\_csv()}\spxextra{in module octapy.tools}}

\begin{fulllineitems}
\phantomsection\label{\detokenize{octapy:octapy.tools.netcdf_to_csv}}\pysiglinewithargsret{\sphinxcode{\sphinxupquote{octapy.tools.}}\sphinxbfcode{\sphinxupquote{netcdf\_to\_csv}}}{\emph{\DUrole{n}{file\_list}}, \emph{\DUrole{n}{outfile}}}{}
\sphinxAtStartPar
convert output netcdf files to a .csv file
\begin{quote}\begin{description}
\item[{Parameters}] \leavevmode\begin{itemize}
\item {} 
\sphinxAtStartPar
\sphinxstyleliteralstrong{\sphinxupquote{file\_list}} \textendash{} a list of paths to the netcdf files to be converted

\item {} 
\sphinxAtStartPar
\sphinxstyleliteralstrong{\sphinxupquote{outfile}} \textendash{} path of the .csv file to be created

\end{itemize}

\item[{Returns}] \leavevmode
\sphinxAtStartPar
None

\end{description}\end{quote}

\end{fulllineitems}

\index{plot\_csv\_output() (in module octapy.tools)@\spxentry{plot\_csv\_output()}\spxextra{in module octapy.tools}}

\begin{fulllineitems}
\phantomsection\label{\detokenize{octapy:octapy.tools.plot_csv_output}}\pysiglinewithargsret{\sphinxcode{\sphinxupquote{octapy.tools.}}\sphinxbfcode{\sphinxupquote{plot\_csv\_output}}}{\emph{\DUrole{n}{file\_list}}, \emph{\DUrole{n}{extent}}, \emph{\DUrole{n}{step}\DUrole{o}{=}\DUrole{default_value}{2}}, \emph{\DUrole{n}{plot\_type}\DUrole{o}{=}\DUrole{default_value}{\textquotesingle{}lines\textquotesingle{}}}, \emph{\DUrole{n}{colors}\DUrole{o}{=}\DUrole{default_value}{None}}}{}
\sphinxAtStartPar
plot output trajectories contained in a .csv file
\begin{quote}\begin{description}
\item[{Parameters}] \leavevmode\begin{itemize}
\item {} 
\sphinxAtStartPar
\sphinxstyleliteralstrong{\sphinxupquote{file\_list}} \textendash{} a list of paths to the .csv files to be plotted

\item {} 
\sphinxAtStartPar
\sphinxstyleliteralstrong{\sphinxupquote{extent}} \textendash{} a list of coordinates representing the extent as {[}minimum longitude,
maximum longitude, minimum latitude, maximum latitude{]}

\end{itemize}

\item[{Returns}] \leavevmode
\sphinxAtStartPar
None

\end{description}\end{quote}

\end{fulllineitems}

\index{plot\_netcdf\_output() (in module octapy.tools)@\spxentry{plot\_netcdf\_output()}\spxextra{in module octapy.tools}}

\begin{fulllineitems}
\phantomsection\label{\detokenize{octapy:octapy.tools.plot_netcdf_output}}\pysiglinewithargsret{\sphinxcode{\sphinxupquote{octapy.tools.}}\sphinxbfcode{\sphinxupquote{plot\_netcdf\_output}}}{\emph{\DUrole{n}{file\_list}}, \emph{\DUrole{n}{extent}}, \emph{\DUrole{n}{out\_file}\DUrole{o}{=}\DUrole{default_value}{None}}, \emph{\DUrole{n}{step}\DUrole{o}{=}\DUrole{default_value}{2}}, \emph{\DUrole{n}{plot\_type}\DUrole{o}{=}\DUrole{default_value}{\textquotesingle{}lines\textquotesingle{}}}, \emph{\DUrole{n}{colors}\DUrole{o}{=}\DUrole{default_value}{None}}, \emph{\DUrole{n}{drifter}\DUrole{o}{=}\DUrole{default_value}{None}}, \emph{\DUrole{n}{contour\_file}\DUrole{o}{=}\DUrole{default_value}{None}}, \emph{\DUrole{n}{contour\_idx}\DUrole{o}{=}\DUrole{default_value}{0}}}{}
\sphinxAtStartPar
plot output trajectories contained in netcdf files
\begin{quote}\begin{description}
\item[{Parameters}] \leavevmode\begin{itemize}
\item {} 
\sphinxAtStartPar
\sphinxstyleliteralstrong{\sphinxupquote{file\_list}} \textendash{} a list of paths to the .csv files to be plotted

\item {} 
\sphinxAtStartPar
\sphinxstyleliteralstrong{\sphinxupquote{extent}} \textendash{} a list of coordinates representing the extent as {[}minimum longitude,
maximum longitude, minimum latitude, maximum latitude{]}

\item {} 
\sphinxAtStartPar
\sphinxstyleliteralstrong{\sphinxupquote{out\_file}} \textendash{} path for the output image file

\item {} 
\sphinxAtStartPar
\sphinxstyleliteralstrong{\sphinxupquote{step}} \textendash{} integer representing the number of trajectory points to plot (e.g,
step=2 means you will show every two points)

\item {} 
\sphinxAtStartPar
\sphinxstyleliteralstrong{\sphinxupquote{plot\_type}} \textendash{} type of plot for each trajectory, may be ‘lines’ or ‘scatter’.
Defaults is ‘lines’.

\item {} 
\sphinxAtStartPar
\sphinxstyleliteralstrong{\sphinxupquote{colors}} \textendash{} list of python colors to customize line colors. Must be same length as
file\_list.

\item {} 
\sphinxAtStartPar
\sphinxstyleliteralstrong{\sphinxupquote{drifter}} \textendash{} path to .csv file containing drifter track containing columns ‘lats’
and ‘lons’

\item {} 
\sphinxAtStartPar
\sphinxstyleliteralstrong{\sphinxupquote{contour\_file}} \textendash{} path to netcdf file containing spatial temperature data

\item {} 
\sphinxAtStartPar
\sphinxstyleliteralstrong{\sphinxupquote{contour\_idx}} \textendash{} index for the sigma level of temperature data to plot (0 = surface)

\end{itemize}

\item[{Returns}] \leavevmode
\sphinxAtStartPar
None

\end{description}\end{quote}

\end{fulllineitems}

\index{build\_skill\_release() (in module octapy.tools)@\spxentry{build\_skill\_release()}\spxextra{in module octapy.tools}}

\begin{fulllineitems}
\phantomsection\label{\detokenize{octapy:octapy.tools.build_skill_release}}\pysiglinewithargsret{\sphinxcode{\sphinxupquote{octapy.tools.}}\sphinxbfcode{\sphinxupquote{build\_skill\_release}}}{\emph{\DUrole{n}{drifter\_file}}, \emph{\DUrole{n}{model}}, \emph{\DUrole{n}{period}\DUrole{o}{=}\DUrole{default_value}{Timedelta(\textquotesingle{}3 days 00:00:00\textquotesingle{})}}, \emph{\DUrole{n}{data\_freq}\DUrole{o}{=}\DUrole{default_value}{Timedelta(\textquotesingle{}0 days 01:00:00\textquotesingle{})}}}{}
\sphinxAtStartPar
Determine the model skill against a particular drifter
\begin{quote}\begin{description}
\item[{Parameters}] \leavevmode\begin{itemize}
\item {} 
\sphinxAtStartPar
\sphinxstyleliteralstrong{\sphinxupquote{drifter\_file}} \textendash{} a .csv file of the drifter data with the columns ‘datetime’, ‘lat’, and
‘lon’

\item {} 
\sphinxAtStartPar
\sphinxstyleliteralstrong{\sphinxupquote{model}} \textendash{} an initialized octapy.tracking.Model object

\item {} 
\sphinxAtStartPar
\sphinxstyleliteralstrong{\sphinxupquote{period}} \textendash{} a pandas Timedelta object representing the time period for which to
calculate the skill

\item {} 
\sphinxAtStartPar
\sphinxstyleliteralstrong{\sphinxupquote{data\_freq}} \textendash{} a pandas Timedelta object representing the frequency of the drifter data

\end{itemize}

\item[{Returns}] \leavevmode
\sphinxAtStartPar
None

\end{description}\end{quote}

\end{fulllineitems}

\index{get\_drifter\_data() (in module octapy.tools)@\spxentry{get\_drifter\_data()}\spxextra{in module octapy.tools}}

\begin{fulllineitems}
\phantomsection\label{\detokenize{octapy:octapy.tools.get_drifter_data}}\pysiglinewithargsret{\sphinxcode{\sphinxupquote{octapy.tools.}}\sphinxbfcode{\sphinxupquote{get\_drifter\_data}}}{\emph{\DUrole{n}{drifter\_file}}, \emph{\DUrole{n}{drifter\_id}}}{}
\sphinxAtStartPar
Gets drifter data from a given .csv file for a particular drifter id
\begin{quote}\begin{description}
\item[{Parameters}] \leavevmode\begin{itemize}
\item {} 
\sphinxAtStartPar
\sphinxstyleliteralstrong{\sphinxupquote{drifter\_file}} \textendash{} a .csv file of the drifter data with the columns ‘datetime’, ‘lat’, and
‘lon’

\item {} 
\sphinxAtStartPar
\sphinxstyleliteralstrong{\sphinxupquote{drifter\_id}} \textendash{} the drifter id with the same type as in the drifter file

\end{itemize}

\item[{Returns}] \leavevmode
\sphinxAtStartPar
a Pandas DataFrame

\end{description}\end{quote}

\end{fulllineitems}

\index{run\_skill\_analysis() (in module octapy.tools)@\spxentry{run\_skill\_analysis()}\spxextra{in module octapy.tools}}

\begin{fulllineitems}
\phantomsection\label{\detokenize{octapy:octapy.tools.run_skill_analysis}}\pysiglinewithargsret{\sphinxcode{\sphinxupquote{octapy.tools.}}\sphinxbfcode{\sphinxupquote{run\_skill\_analysis}}}{\emph{\DUrole{n}{drifter\_file}}, \emph{\DUrole{n}{drifter\_id}}, \emph{\DUrole{n}{skill\_files}}, \emph{\DUrole{n}{date\_range}}, \emph{\DUrole{n}{grid}}, \emph{\DUrole{n}{period}\DUrole{o}{=}\DUrole{default_value}{Timedelta(\textquotesingle{}3 days 00:00:00\textquotesingle{})}}, \emph{\DUrole{n}{data\_freq}\DUrole{o}{=}\DUrole{default_value}{Timedelta(\textquotesingle{}0 days 01:00:00\textquotesingle{})}}}{}
\sphinxAtStartPar
Calculate the skill score for a given particle run. Here, the skill score
is calculated as in \sphinxhref{https://doi.org/10.1029/2010JC006837}{Liu and Weisberg (2011)}.
\begin{quote}\begin{description}
\item[{Parameters}] \leavevmode\begin{itemize}
\item {} 
\sphinxAtStartPar
\sphinxstyleliteralstrong{\sphinxupquote{drifter\_file}} \textendash{} a .csv file of the drifter data with the columns ‘datetime’, ‘lat’, and
‘lon’

\item {} 
\sphinxAtStartPar
\sphinxstyleliteralstrong{\sphinxupquote{skill\_files}} \textendash{} a sorted list of the model output netCDF files that
contain the tracks from the model runs for skill

\item {} 
\sphinxAtStartPar
\sphinxstyleliteralstrong{\sphinxupquote{date\_range}} \textendash{} a numpy.ndarray object of numpy.datetime64 objects, the drifter and
output data frequencies should match

\item {} 
\sphinxAtStartPar
\sphinxstyleliteralstrong{\sphinxupquote{period}} \textendash{} a pandas Timedelta object representing the time period for which to
calculate the skill

\item {} 
\sphinxAtStartPar
\sphinxstyleliteralstrong{\sphinxupquote{data\_freq}} \textendash{} a pandas Timedelta object representing the frequency of the drifter data

\end{itemize}

\item[{Returns}] \leavevmode
\sphinxAtStartPar
a numpy array of particle times, trajectory lengths, separation
distances, and skill scores

\end{description}\end{quote}

\end{fulllineitems}



\chapter{octapy.tracking module}
\label{\detokenize{octapy:module-octapy.tracking}}\label{\detokenize{octapy:octapy-tracking-module}}\index{module@\spxentry{module}!octapy.tracking@\spxentry{octapy.tracking}}\index{octapy.tracking@\spxentry{octapy.tracking}!module@\spxentry{module}}\index{Model (class in octapy.tracking)@\spxentry{Model}\spxextra{class in octapy.tracking}}

\begin{fulllineitems}
\phantomsection\label{\detokenize{octapy:octapy.tracking.Model}}\pysiglinewithargsret{\sphinxbfcode{\sphinxupquote{class }}\sphinxcode{\sphinxupquote{octapy.tracking.}}\sphinxbfcode{\sphinxupquote{Model}}}{\emph{\DUrole{n}{release\_file}\DUrole{o}{=}\DUrole{default_value}{None}}, \emph{\DUrole{n}{model}\DUrole{o}{=}\DUrole{default_value}{None}}, \emph{\DUrole{n}{submodel}\DUrole{o}{=}\DUrole{default_value}{None}}, \emph{\DUrole{n}{data\_dir}\DUrole{o}{=}\DUrole{default_value}{\textquotesingle{}data\textquotesingle{}}}, \emph{\DUrole{n}{direction}\DUrole{o}{=}\DUrole{default_value}{1}}, \emph{\DUrole{n}{dims}\DUrole{o}{=}\DUrole{default_value}{2}}, \emph{\DUrole{n}{diffusion}\DUrole{o}{=}\DUrole{default_value}{False}}, \emph{\DUrole{n}{depth}\DUrole{o}{=}\DUrole{default_value}{None}}, \emph{\DUrole{n}{extent}\DUrole{o}{=}\DUrole{default_value}{None}}, \emph{\DUrole{n}{data\_date\_range}\DUrole{o}{=}\DUrole{default_value}{None}}, \emph{\DUrole{n}{timestep}\DUrole{o}{=}\DUrole{default_value}{numpy.timedelta64(60, \textquotesingle{}m\textquotesingle{})}}, \emph{\DUrole{n}{data\_freq}\DUrole{o}{=}\DUrole{default_value}{numpy.timedelta64(60, \textquotesingle{}m\textquotesingle{})}}, \emph{\DUrole{n}{data\_timestep}\DUrole{o}{=}\DUrole{default_value}{numpy.timedelta64(60, \textquotesingle{}m\textquotesingle{})}}, \emph{\DUrole{n}{interp}\DUrole{o}{=}\DUrole{default_value}{\textquotesingle{}linear\textquotesingle{}}}, \emph{\DUrole{n}{leafsize}\DUrole{o}{=}\DUrole{default_value}{9}}, \emph{\DUrole{n}{vert\_migration}\DUrole{o}{=}\DUrole{default_value}{False}}, \emph{\DUrole{n}{vert\_array}\DUrole{o}{=}\DUrole{default_value}{None}}, \emph{\DUrole{n}{output\_file}\DUrole{o}{=}\DUrole{default_value}{None}}, \emph{\DUrole{n}{output\_freq}\DUrole{o}{=}\DUrole{default_value}{numpy.timedelta64(60, \textquotesingle{}m\textquotesingle{})}}}{}
\sphinxAtStartPar
Bases: \sphinxcode{\sphinxupquote{object}}

\sphinxAtStartPar
A particle tracking Model object
\begin{quote}\begin{description}
\item[{Parameters}] \leavevmode\begin{itemize}
\item {} 
\sphinxAtStartPar
\sphinxstyleliteralstrong{\sphinxupquote{release\_file}} \textendash{} a file containing the particle coordinates and release times

\item {} 
\sphinxAtStartPar
\sphinxstyleliteralstrong{\sphinxupquote{model}} \textendash{} name string of the ocean model used for input data (e.g, ‘HYCOM’)

\item {} 
\sphinxAtStartPar
\sphinxstyleliteralstrong{\sphinxupquote{submodel}} \textendash{} name string of the submodel and/or experiment used for input data
(e.g, ‘GOMl0.04/expt\_31.0’)

\item {} 
\sphinxAtStartPar
\sphinxstyleliteralstrong{\sphinxupquote{data\_dir}} \textendash{} data directory path

\item {} 
\sphinxAtStartPar
\sphinxstyleliteralstrong{\sphinxupquote{direction}} \textendash{} \begin{description}
\item[{forcing direction through time. Must be  1 for forward or \sphinxhyphen{}1}] \leavevmode
\sphinxAtStartPar
for backward

\end{description}


\item {} 
\sphinxAtStartPar
\sphinxstyleliteralstrong{\sphinxupquote{dims}} \textendash{} dimensionality of the model, must be 2 or 3

\item {} 
\sphinxAtStartPar
\sphinxstyleliteralstrong{\sphinxupquote{diffusion}} \textendash{} if true, enables diffusion

\item {} 
\sphinxAtStartPar
\sphinxstyleliteralstrong{\sphinxupquote{data\_vars}} \textendash{} an numpy.ndarray of the variable names in the data file

\item {} 
\sphinxAtStartPar
\sphinxstyleliteralstrong{\sphinxupquote{depth}} \textendash{} forcing depth if the model is 2\sphinxhyphen{}dimensional

\item {} 
\sphinxAtStartPar
\sphinxstyleliteralstrong{\sphinxupquote{extent}} \textendash{} a list of extent coordinates as {[}minlat, maxlat, minlon, maxlon{]}

\item {} 
\sphinxAtStartPar
\sphinxstyleliteralstrong{\sphinxupquote{data\_date\_range}} \textendash{} a numpy.ndarray object of numpy.datetime64 objects containing the dates
of the input data. WARNING: Must match frequency of data

\item {} 
\sphinxAtStartPar
\sphinxstyleliteralstrong{\sphinxupquote{timestep}} \textendash{} a numpy.timedelta64 object representing the timestep of the tracking
model in minutes (e.g., np.timedelta64(60,’m’))

\item {} 
\sphinxAtStartPar
\sphinxstyleliteralstrong{\sphinxupquote{data\_freq}} \textendash{} a numpy.timedelta64 object representing the frequency of data on the
data server (e.g., np.timedelta64(60,’m’))

\item {} 
\sphinxAtStartPar
\sphinxstyleliteralstrong{\sphinxupquote{data\_timestep}} \textendash{} a numpy.timedelta64 object representing the timestep of the data to be
downloaded in minutes (e.g., np.timedelta64(60,’m’))

\item {} 
\sphinxAtStartPar
\sphinxstyleliteralstrong{\sphinxupquote{interp}} \textendash{} interpolation method. Supported are ‘linear’, ‘nearest’, ‘splinef2d’,
and ‘idw’.

\item {} 
\sphinxAtStartPar
\sphinxstyleliteralstrong{\sphinxupquote{leafsize}} \textendash{} number of nearest neighbors for some interpolation schemes

\item {} 
\sphinxAtStartPar
\sphinxstyleliteralstrong{\sphinxupquote{vert\_migration}} \textendash{} if True, the particle will undergo daily vertical migrations which will
override w velocities

\item {} 
\sphinxAtStartPar
\sphinxstyleliteralstrong{\sphinxupquote{vert\_array}} \textendash{} an array of length 24 representing the depth of a particle over a
24\sphinxhyphen{}hour period when vert\_mirgration is set to True

\item {} 
\sphinxAtStartPar
\sphinxstyleliteralstrong{\sphinxupquote{output\_file}} \textendash{} base output file name

\item {} 
\sphinxAtStartPar
\sphinxstyleliteralstrong{\sphinxupquote{output\_freq}} \textendash{} a numpy.timedelta64 object representing how often the particle data will
be output to the output file in minutes (e.g., np.timedelta64(60,’m’))

\end{itemize}

\item[{Returns}] \leavevmode
\sphinxAtStartPar
A Model object

\end{description}\end{quote}
\index{\_\_init\_\_() (octapy.tracking.Model method)@\spxentry{\_\_init\_\_()}\spxextra{octapy.tracking.Model method}}

\begin{fulllineitems}
\phantomsection\label{\detokenize{octapy:octapy.tracking.Model.__init__}}\pysiglinewithargsret{\sphinxbfcode{\sphinxupquote{\_\_init\_\_}}}{\emph{\DUrole{n}{release\_file}\DUrole{o}{=}\DUrole{default_value}{None}}, \emph{\DUrole{n}{model}\DUrole{o}{=}\DUrole{default_value}{None}}, \emph{\DUrole{n}{submodel}\DUrole{o}{=}\DUrole{default_value}{None}}, \emph{\DUrole{n}{data\_dir}\DUrole{o}{=}\DUrole{default_value}{\textquotesingle{}data\textquotesingle{}}}, \emph{\DUrole{n}{direction}\DUrole{o}{=}\DUrole{default_value}{1}}, \emph{\DUrole{n}{dims}\DUrole{o}{=}\DUrole{default_value}{2}}, \emph{\DUrole{n}{diffusion}\DUrole{o}{=}\DUrole{default_value}{False}}, \emph{\DUrole{n}{depth}\DUrole{o}{=}\DUrole{default_value}{None}}, \emph{\DUrole{n}{extent}\DUrole{o}{=}\DUrole{default_value}{None}}, \emph{\DUrole{n}{data\_date\_range}\DUrole{o}{=}\DUrole{default_value}{None}}, \emph{\DUrole{n}{timestep}\DUrole{o}{=}\DUrole{default_value}{numpy.timedelta64(60, \textquotesingle{}m\textquotesingle{})}}, \emph{\DUrole{n}{data\_freq}\DUrole{o}{=}\DUrole{default_value}{numpy.timedelta64(60, \textquotesingle{}m\textquotesingle{})}}, \emph{\DUrole{n}{data\_timestep}\DUrole{o}{=}\DUrole{default_value}{numpy.timedelta64(60, \textquotesingle{}m\textquotesingle{})}}, \emph{\DUrole{n}{interp}\DUrole{o}{=}\DUrole{default_value}{\textquotesingle{}linear\textquotesingle{}}}, \emph{\DUrole{n}{leafsize}\DUrole{o}{=}\DUrole{default_value}{9}}, \emph{\DUrole{n}{vert\_migration}\DUrole{o}{=}\DUrole{default_value}{False}}, \emph{\DUrole{n}{vert\_array}\DUrole{o}{=}\DUrole{default_value}{None}}, \emph{\DUrole{n}{output\_file}\DUrole{o}{=}\DUrole{default_value}{None}}, \emph{\DUrole{n}{output\_freq}\DUrole{o}{=}\DUrole{default_value}{numpy.timedelta64(60, \textquotesingle{}m\textquotesingle{})}}}{}
\sphinxAtStartPar
Initialize self.  See help(type(self)) for accurate signature.

\end{fulllineitems}


\end{fulllineitems}

\index{Grid (class in octapy.tracking)@\spxentry{Grid}\spxextra{class in octapy.tracking}}

\begin{fulllineitems}
\phantomsection\label{\detokenize{octapy:octapy.tracking.Grid}}\pysiglinewithargsret{\sphinxbfcode{\sphinxupquote{class }}\sphinxcode{\sphinxupquote{octapy.tracking.}}\sphinxbfcode{\sphinxupquote{Grid}}}{\emph{\DUrole{n}{model}}}{}
\sphinxAtStartPar
Bases: \sphinxcode{\sphinxupquote{object}}

\sphinxAtStartPar
A Grid object. You must have already downloaded the data into the data
directory.
\begin{quote}\begin{description}
\item[{Parameters}] \leavevmode
\sphinxAtStartPar
\sphinxstyleliteralstrong{\sphinxupquote{model}} \textendash{} Model instance to which the Grid instance will belong

\item[{Returns}] \leavevmode
\sphinxAtStartPar
A Grid object

\end{description}\end{quote}
\index{\_\_init\_\_() (octapy.tracking.Grid method)@\spxentry{\_\_init\_\_()}\spxextra{octapy.tracking.Grid method}}

\begin{fulllineitems}
\phantomsection\label{\detokenize{octapy:octapy.tracking.Grid.__init__}}\pysiglinewithargsret{\sphinxbfcode{\sphinxupquote{\_\_init\_\_}}}{\emph{\DUrole{n}{model}}}{}
\sphinxAtStartPar
Initialize self.  See help(type(self)) for accurate signature.

\end{fulllineitems}

\index{src\_crs (octapy.tracking.Grid attribute)@\spxentry{src\_crs}\spxextra{octapy.tracking.Grid attribute}}

\begin{fulllineitems}
\phantomsection\label{\detokenize{octapy:octapy.tracking.Grid.src_crs}}\pysigline{\sphinxbfcode{\sphinxupquote{src\_crs}}}
\sphinxAtStartPar
a cartopy.crs projection object representing the data’s source 
projection (e.g., cartopy.crs.LambertCylindrical())

\end{fulllineitems}

\index{tgt\_crs (octapy.tracking.Grid attribute)@\spxentry{tgt\_crs}\spxextra{octapy.tracking.Grid attribute}}

\begin{fulllineitems}
\phantomsection\label{\detokenize{octapy:octapy.tracking.Grid.tgt_crs}}\pysigline{\sphinxbfcode{\sphinxupquote{tgt\_crs}}}
\sphinxAtStartPar
a cartopy.crs projection object representing the model’s target 
projection

\end{fulllineitems}

\index{file (octapy.tracking.Grid attribute)@\spxentry{file}\spxextra{octapy.tracking.Grid attribute}}

\begin{fulllineitems}
\phantomsection\label{\detokenize{octapy:octapy.tracking.Grid.file}}\pysigline{\sphinxbfcode{\sphinxupquote{file}}}
\sphinxAtStartPar
name of the file from which the grid was produced.

\end{fulllineitems}

\index{depths (octapy.tracking.Grid attribute)@\spxentry{depths}\spxextra{octapy.tracking.Grid attribute}}

\begin{fulllineitems}
\phantomsection\label{\detokenize{octapy:octapy.tracking.Grid.depths}}\pysigline{\sphinxbfcode{\sphinxupquote{depths}}}
\sphinxAtStartPar
depths of the grid

\end{fulllineitems}

\index{lons (octapy.tracking.Grid attribute)@\spxentry{lons}\spxextra{octapy.tracking.Grid attribute}}

\begin{fulllineitems}
\phantomsection\label{\detokenize{octapy:octapy.tracking.Grid.lons}}\pysigline{\sphinxbfcode{\sphinxupquote{lons}}}
\sphinxAtStartPar
longitudes of the grid

\end{fulllineitems}

\index{lats (octapy.tracking.Grid attribute)@\spxentry{lats}\spxextra{octapy.tracking.Grid attribute}}

\begin{fulllineitems}
\phantomsection\label{\detokenize{octapy:octapy.tracking.Grid.lats}}\pysigline{\sphinxbfcode{\sphinxupquote{lats}}}
\sphinxAtStartPar
latitudes of the grid

\end{fulllineitems}

\index{x (octapy.tracking.Grid attribute)@\spxentry{x}\spxextra{octapy.tracking.Grid attribute}}

\begin{fulllineitems}
\phantomsection\label{\detokenize{octapy:octapy.tracking.Grid.x}}\pysigline{\sphinxbfcode{\sphinxupquote{x}}}
\sphinxAtStartPar
the x coordinates of the grid in meters

\end{fulllineitems}

\index{y (octapy.tracking.Grid attribute)@\spxentry{y}\spxextra{octapy.tracking.Grid attribute}}

\begin{fulllineitems}
\phantomsection\label{\detokenize{octapy:octapy.tracking.Grid.y}}\pysigline{\sphinxbfcode{\sphinxupquote{y}}}
\sphinxAtStartPar
the y coordinates of the grid in meters

\end{fulllineitems}

\index{points (octapy.tracking.Grid attribute)@\spxentry{points}\spxextra{octapy.tracking.Grid attribute}}

\begin{fulllineitems}
\phantomsection\label{\detokenize{octapy:octapy.tracking.Grid.points}}\pysigline{\sphinxbfcode{\sphinxupquote{points}}}
\sphinxAtStartPar
an array of grid coordinates in meters

\end{fulllineitems}

\index{tree (octapy.tracking.Grid attribute)@\spxentry{tree}\spxextra{octapy.tracking.Grid attribute}}

\begin{fulllineitems}
\phantomsection\label{\detokenize{octapy:octapy.tracking.Grid.tree}}\pysigline{\sphinxbfcode{\sphinxupquote{tree}}}
\sphinxAtStartPar
a scipy.spatial.cKDTree instance representing the grid

\end{fulllineitems}


\end{fulllineitems}

\index{Particle (class in octapy.tracking)@\spxentry{Particle}\spxextra{class in octapy.tracking}}

\begin{fulllineitems}
\phantomsection\label{\detokenize{octapy:octapy.tracking.Particle}}\pysiglinewithargsret{\sphinxbfcode{\sphinxupquote{class }}\sphinxcode{\sphinxupquote{octapy.tracking.}}\sphinxbfcode{\sphinxupquote{Particle}}}{\emph{\DUrole{n}{num}\DUrole{o}{=}\DUrole{default_value}{None}}, \emph{\DUrole{n}{lat}\DUrole{o}{=}\DUrole{default_value}{None}}, \emph{\DUrole{n}{lon}\DUrole{o}{=}\DUrole{default_value}{None}}, \emph{\DUrole{n}{depth}\DUrole{o}{=}\DUrole{default_value}{None}}, \emph{\DUrole{n}{timestamp}\DUrole{o}{=}\DUrole{default_value}{None}}, \emph{\DUrole{n}{x}\DUrole{o}{=}\DUrole{default_value}{None}}, \emph{\DUrole{n}{y}\DUrole{o}{=}\DUrole{default_value}{None}}, \emph{\DUrole{n}{u}\DUrole{o}{=}\DUrole{default_value}{None}}, \emph{\DUrole{n}{v}\DUrole{o}{=}\DUrole{default_value}{None}}, \emph{\DUrole{n}{w}\DUrole{o}{=}\DUrole{default_value}{None}}, \emph{\DUrole{n}{temp}\DUrole{o}{=}\DUrole{default_value}{None}}, \emph{\DUrole{n}{sal}\DUrole{o}{=}\DUrole{default_value}{None}}, \emph{\DUrole{n}{ssh}\DUrole{o}{=}\DUrole{default_value}{None}}, \emph{\DUrole{n}{filepath}\DUrole{o}{=}\DUrole{default_value}{None}}}{}
\sphinxAtStartPar
Bases: \sphinxcode{\sphinxupquote{object}}

\sphinxAtStartPar
A Particle object
\begin{quote}\begin{description}
\item[{Parameters}] \leavevmode\begin{itemize}
\item {} 
\sphinxAtStartPar
\sphinxstyleliteralstrong{\sphinxupquote{lat}} \textendash{} Latitude of the particle

\item {} 
\sphinxAtStartPar
\sphinxstyleliteralstrong{\sphinxupquote{lon}} \textendash{} Longitude of the particle

\item {} 
\sphinxAtStartPar
\sphinxstyleliteralstrong{\sphinxupquote{depth}} \textendash{} Depth of the particle in meters

\item {} 
\sphinxAtStartPar
\sphinxstyleliteralstrong{\sphinxupquote{timestamp}} \textendash{} a numpy.datetime64 instance representing the time of the particle

\end{itemize}

\item[{Returns}] \leavevmode
\sphinxAtStartPar
A Particle object

\end{description}\end{quote}
\index{\_\_init\_\_() (octapy.tracking.Particle method)@\spxentry{\_\_init\_\_()}\spxextra{octapy.tracking.Particle method}}

\begin{fulllineitems}
\phantomsection\label{\detokenize{octapy:octapy.tracking.Particle.__init__}}\pysiglinewithargsret{\sphinxbfcode{\sphinxupquote{\_\_init\_\_}}}{\emph{\DUrole{n}{num}\DUrole{o}{=}\DUrole{default_value}{None}}, \emph{\DUrole{n}{lat}\DUrole{o}{=}\DUrole{default_value}{None}}, \emph{\DUrole{n}{lon}\DUrole{o}{=}\DUrole{default_value}{None}}, \emph{\DUrole{n}{depth}\DUrole{o}{=}\DUrole{default_value}{None}}, \emph{\DUrole{n}{timestamp}\DUrole{o}{=}\DUrole{default_value}{None}}, \emph{\DUrole{n}{x}\DUrole{o}{=}\DUrole{default_value}{None}}, \emph{\DUrole{n}{y}\DUrole{o}{=}\DUrole{default_value}{None}}, \emph{\DUrole{n}{u}\DUrole{o}{=}\DUrole{default_value}{None}}, \emph{\DUrole{n}{v}\DUrole{o}{=}\DUrole{default_value}{None}}, \emph{\DUrole{n}{w}\DUrole{o}{=}\DUrole{default_value}{None}}, \emph{\DUrole{n}{temp}\DUrole{o}{=}\DUrole{default_value}{None}}, \emph{\DUrole{n}{sal}\DUrole{o}{=}\DUrole{default_value}{None}}, \emph{\DUrole{n}{ssh}\DUrole{o}{=}\DUrole{default_value}{None}}, \emph{\DUrole{n}{filepath}\DUrole{o}{=}\DUrole{default_value}{None}}}{}
\sphinxAtStartPar
Initialize self.  See help(type(self)) for accurate signature.

\end{fulllineitems}

\index{x (octapy.tracking.Particle attribute)@\spxentry{x}\spxextra{octapy.tracking.Particle attribute}}

\begin{fulllineitems}
\phantomsection\label{\detokenize{octapy:octapy.tracking.Particle.x}}\pysigline{\sphinxbfcode{\sphinxupquote{x}}}
\sphinxAtStartPar
the x coordinate in meters

\end{fulllineitems}

\index{y (octapy.tracking.Particle attribute)@\spxentry{y}\spxextra{octapy.tracking.Particle attribute}}

\begin{fulllineitems}
\phantomsection\label{\detokenize{octapy:octapy.tracking.Particle.y}}\pysigline{\sphinxbfcode{\sphinxupquote{y}}}
\sphinxAtStartPar
the y coordinate in meters

\end{fulllineitems}

\index{u (octapy.tracking.Particle attribute)@\spxentry{u}\spxextra{octapy.tracking.Particle attribute}}

\begin{fulllineitems}
\phantomsection\label{\detokenize{octapy:octapy.tracking.Particle.u}}\pysigline{\sphinxbfcode{\sphinxupquote{u}}}
\sphinxAtStartPar
the u\sphinxhyphen{}velocity (eastward velocity) at the particle’s location

\end{fulllineitems}

\index{v (octapy.tracking.Particle attribute)@\spxentry{v}\spxextra{octapy.tracking.Particle attribute}}

\begin{fulllineitems}
\phantomsection\label{\detokenize{octapy:octapy.tracking.Particle.v}}\pysigline{\sphinxbfcode{\sphinxupquote{v}}}
\sphinxAtStartPar
the v\sphinxhyphen{}velocity (northward velocity) at the particle’s location

\end{fulllineitems}

\index{w (octapy.tracking.Particle attribute)@\spxentry{w}\spxextra{octapy.tracking.Particle attribute}}

\begin{fulllineitems}
\phantomsection\label{\detokenize{octapy:octapy.tracking.Particle.w}}\pysigline{\sphinxbfcode{\sphinxupquote{w}}}
\sphinxAtStartPar
the w\sphinxhyphen{}velocity (upward velocity) at the particle’s location

\end{fulllineitems}

\index{temp (octapy.tracking.Particle attribute)@\spxentry{temp}\spxextra{octapy.tracking.Particle attribute}}

\begin{fulllineitems}
\phantomsection\label{\detokenize{octapy:octapy.tracking.Particle.temp}}\pysigline{\sphinxbfcode{\sphinxupquote{temp}}}
\sphinxAtStartPar
the temperature at the particle’s location

\end{fulllineitems}

\index{sal (octapy.tracking.Particle attribute)@\spxentry{sal}\spxextra{octapy.tracking.Particle attribute}}

\begin{fulllineitems}
\phantomsection\label{\detokenize{octapy:octapy.tracking.Particle.sal}}\pysigline{\sphinxbfcode{\sphinxupquote{sal}}}
\sphinxAtStartPar
the salinity at the particle’s location

\end{fulllineitems}

\index{ssh (octapy.tracking.Particle attribute)@\spxentry{ssh}\spxextra{octapy.tracking.Particle attribute}}

\begin{fulllineitems}
\phantomsection\label{\detokenize{octapy:octapy.tracking.Particle.ssh}}\pysigline{\sphinxbfcode{\sphinxupquote{ssh}}}
\sphinxAtStartPar
the sea surface height at the particle’s location

\end{fulllineitems}

\index{filepath (octapy.tracking.Particle attribute)@\spxentry{filepath}\spxextra{octapy.tracking.Particle attribute}}

\begin{fulllineitems}
\phantomsection\label{\detokenize{octapy:octapy.tracking.Particle.filepath}}\pysigline{\sphinxbfcode{\sphinxupquote{filepath}}}
\sphinxAtStartPar
the expected file path given the particles timestamp

\end{fulllineitems}


\end{fulllineitems}

\index{deepcopy() (in module octapy.tracking)@\spxentry{deepcopy()}\spxextra{in module octapy.tracking}}

\begin{fulllineitems}
\phantomsection\label{\detokenize{octapy:octapy.tracking.deepcopy}}\pysiglinewithargsret{\sphinxcode{\sphinxupquote{octapy.tracking.}}\sphinxbfcode{\sphinxupquote{deepcopy}}}{\emph{\DUrole{n}{particle}}}{}
\sphinxAtStartPar
make a deep copy of a particle
\begin{quote}\begin{description}
\item[{Parameters}] \leavevmode
\sphinxAtStartPar
\sphinxstyleliteralstrong{\sphinxupquote{particle}} \textendash{} A particle instance

\item[{Returns}] \leavevmode
\sphinxAtStartPar
A Particle object

\end{description}\end{quote}

\end{fulllineitems}

\index{transform() (in module octapy.tracking)@\spxentry{transform()}\spxextra{in module octapy.tracking}}

\begin{fulllineitems}
\phantomsection\label{\detokenize{octapy:octapy.tracking.transform}}\pysiglinewithargsret{\sphinxcode{\sphinxupquote{octapy.tracking.}}\sphinxbfcode{\sphinxupquote{transform}}}{\emph{\DUrole{n}{src\_crs}}, \emph{\DUrole{n}{tgt\_crs}}, \emph{\DUrole{n}{lon}}, \emph{\DUrole{n}{lat}}}{}
\sphinxAtStartPar
transform the longitude and latitude of a point to x and y coordinates
for a given coordinate reference system
\begin{quote}\begin{description}
\item[{Parameters}] \leavevmode\begin{itemize}
\item {} 
\sphinxAtStartPar
\sphinxstyleliteralstrong{\sphinxupquote{src\_crs}} \textendash{} the source coordinate reference system

\item {} 
\sphinxAtStartPar
\sphinxstyleliteralstrong{\sphinxupquote{tgt\_crs}} \textendash{} the target coordinate reference system

\item {} 
\sphinxAtStartPar
\sphinxstyleliteralstrong{\sphinxupquote{lon}} \textendash{} the longitude of the point

\item {} 
\sphinxAtStartPar
\sphinxstyleliteralstrong{\sphinxupquote{lat}} \textendash{} the latitude of the point

\end{itemize}

\item[{Returns}] \leavevmode
\sphinxAtStartPar
the x and y coordinates of the point

\end{description}\end{quote}

\end{fulllineitems}

\index{download\_hycom\_data() (in module octapy.tracking)@\spxentry{download\_hycom\_data()}\spxextra{in module octapy.tracking}}

\begin{fulllineitems}
\phantomsection\label{\detokenize{octapy:octapy.tracking.download_hycom_data}}\pysiglinewithargsret{\sphinxcode{\sphinxupquote{octapy.tracking.}}\sphinxbfcode{\sphinxupquote{download\_hycom\_data}}}{\emph{\DUrole{n}{model}}}{}
\sphinxAtStartPar
Download the input data for a given Model instance
\begin{quote}\begin{description}
\item[{Parameters}] \leavevmode
\sphinxAtStartPar
\sphinxstyleliteralstrong{\sphinxupquote{model}} \textendash{} a model instance

\item[{Returns}] \leavevmode
\sphinxAtStartPar
None

\end{description}\end{quote}

\end{fulllineitems}

\index{get\_physical() (in module octapy.tracking)@\spxentry{get\_physical()}\spxextra{in module octapy.tracking}}

\begin{fulllineitems}
\phantomsection\label{\detokenize{octapy:octapy.tracking.get_physical}}\pysiglinewithargsret{\sphinxcode{\sphinxupquote{octapy.tracking.}}\sphinxbfcode{\sphinxupquote{get\_physical}}}{\emph{\DUrole{n}{particle}}, \emph{\DUrole{n}{grid}}, \emph{\DUrole{n}{model}}}{}
\sphinxAtStartPar
wrapper for the interp3d function that interpolates for time if
necessary
\begin{quote}\begin{description}
\item[{Parameters}] \leavevmode\begin{itemize}
\item {} 
\sphinxAtStartPar
\sphinxstyleliteralstrong{\sphinxupquote{particle}} \textendash{} a particle instance

\item {} 
\sphinxAtStartPar
\sphinxstyleliteralstrong{\sphinxupquote{grid}} \textendash{} a grid instance

\item {} 
\sphinxAtStartPar
\sphinxstyleliteralstrong{\sphinxupquote{model}} \textendash{} a model instance

\end{itemize}

\item[{Returns}] \leavevmode
\sphinxAtStartPar
a particle instance

\end{description}\end{quote}

\end{fulllineitems}

\index{interp3d() (in module octapy.tracking)@\spxentry{interp3d()}\spxextra{in module octapy.tracking}}

\begin{fulllineitems}
\phantomsection\label{\detokenize{octapy:octapy.tracking.interp3d}}\pysiglinewithargsret{\sphinxcode{\sphinxupquote{octapy.tracking.}}\sphinxbfcode{\sphinxupquote{interp3d}}}{\emph{\DUrole{n}{particle}}, \emph{\DUrole{n}{grid}}, \emph{\DUrole{n}{model}}, \emph{\DUrole{n}{power}\DUrole{o}{=}\DUrole{default_value}{1.0}}}{}
\sphinxAtStartPar
fills the interpolated environmental attributes for a particle instance
\begin{quote}\begin{description}
\item[{Parameters}] \leavevmode\begin{itemize}
\item {} 
\sphinxAtStartPar
\sphinxstyleliteralstrong{\sphinxupquote{particle}} \textendash{} a particle instance

\item {} 
\sphinxAtStartPar
\sphinxstyleliteralstrong{\sphinxupquote{grid}} \textendash{} a grid instance

\item {} 
\sphinxAtStartPar
\sphinxstyleliteralstrong{\sphinxupquote{model}} \textendash{} a model instance

\item {} 
\sphinxAtStartPar
\sphinxstyleliteralstrong{\sphinxupquote{power}} \textendash{} the power used in the inverse distance weighted interpolation

\end{itemize}

\item[{Returns}] \leavevmode
\sphinxAtStartPar
a particle instance

\end{description}\end{quote}

\end{fulllineitems}

\index{interp\_for\_time() (in module octapy.tracking)@\spxentry{interp\_for\_time()}\spxextra{in module octapy.tracking}}

\begin{fulllineitems}
\phantomsection\label{\detokenize{octapy:octapy.tracking.interp_for_time}}\pysiglinewithargsret{\sphinxcode{\sphinxupquote{octapy.tracking.}}\sphinxbfcode{\sphinxupquote{interp\_for\_time}}}{\emph{\DUrole{n}{particle}}, \emph{\DUrole{n}{particle1}}, \emph{\DUrole{n}{particle2}}, \emph{\DUrole{n}{dims}\DUrole{o}{=}\DUrole{default_value}{2}}}{}
\sphinxAtStartPar
linear interpolation of the environmental parameters for a particle in
time
\begin{quote}\begin{description}
\item[{Parameters}] \leavevmode\begin{itemize}
\item {} 
\sphinxAtStartPar
\sphinxstyleliteralstrong{\sphinxupquote{particle}} \textendash{} a Particle instance at an invalid timestep for which environmental
parameters are being interpolated in time

\item {} 
\sphinxAtStartPar
\sphinxstyleliteralstrong{\sphinxupquote{particle1}} \textendash{} a Particle instance at the last valid data timestep

\item {} 
\sphinxAtStartPar
\sphinxstyleliteralstrong{\sphinxupquote{particle2}} \textendash{} a Particle instance at the next valid data timestep

\item {} 
\sphinxAtStartPar
\sphinxstyleliteralstrong{\sphinxupquote{dims}} \textendash{} dimensions of the Model instance

\end{itemize}

\item[{Returns}] \leavevmode
\sphinxAtStartPar
a Particle instance

\end{description}\end{quote}

\end{fulllineitems}

\index{force\_particle() (in module octapy.tracking)@\spxentry{force\_particle()}\spxextra{in module octapy.tracking}}

\begin{fulllineitems}
\phantomsection\label{\detokenize{octapy:octapy.tracking.force_particle}}\pysiglinewithargsret{\sphinxcode{\sphinxupquote{octapy.tracking.}}\sphinxbfcode{\sphinxupquote{force\_particle}}}{\emph{\DUrole{n}{particle}}, \emph{\DUrole{n}{grid}}, \emph{\DUrole{n}{model}}}{}
\sphinxAtStartPar
force the particle to the next timestep
\begin{quote}\begin{description}
\item[{Parameters}] \leavevmode\begin{itemize}
\item {} 
\sphinxAtStartPar
\sphinxstyleliteralstrong{\sphinxupquote{particle}} \textendash{} a particle instance

\item {} 
\sphinxAtStartPar
\sphinxstyleliteralstrong{\sphinxupquote{grid}} \textendash{} a grid instance

\item {} 
\sphinxAtStartPar
\sphinxstyleliteralstrong{\sphinxupquote{model}} \textendash{} a model instance

\end{itemize}

\item[{Returns}] \leavevmode
\sphinxAtStartPar
a particle instance

\end{description}\end{quote}

\end{fulllineitems}

\index{add\_row\_to\_arr() (in module octapy.tracking)@\spxentry{add\_row\_to\_arr()}\spxextra{in module octapy.tracking}}

\begin{fulllineitems}
\phantomsection\label{\detokenize{octapy:octapy.tracking.add_row_to_arr}}\pysiglinewithargsret{\sphinxcode{\sphinxupquote{octapy.tracking.}}\sphinxbfcode{\sphinxupquote{add\_row\_to\_arr}}}{\emph{\DUrole{n}{arr}}, \emph{\DUrole{n}{particle}}}{}
\sphinxAtStartPar
add a row containing the Particle attributes to an array
\begin{quote}\begin{description}
\item[{Parameters}] \leavevmode\begin{itemize}
\item {} 
\sphinxAtStartPar
\sphinxstyleliteralstrong{\sphinxupquote{arr}} \textendash{} a NumPy array

\item {} 
\sphinxAtStartPar
\sphinxstyleliteralstrong{\sphinxupquote{particle}} \textendash{} a Particle instance

\end{itemize}

\item[{Returns}] \leavevmode
\sphinxAtStartPar
a NumPy array

\end{description}\end{quote}

\end{fulllineitems}

\index{run\_2d\_model() (in module octapy.tracking)@\spxentry{run\_2d\_model()}\spxextra{in module octapy.tracking}}

\begin{fulllineitems}
\phantomsection\label{\detokenize{octapy:octapy.tracking.run_2d_model}}\pysiglinewithargsret{\sphinxcode{\sphinxupquote{octapy.tracking.}}\sphinxbfcode{\sphinxupquote{run\_2d\_model}}}{\emph{\DUrole{n}{model}}, \emph{\DUrole{n}{grid}}}{}
\sphinxAtStartPar
generate the trajectories for a 2D Model instance
\begin{quote}\begin{description}
\item[{Parameters}] \leavevmode\begin{itemize}
\item {} 
\sphinxAtStartPar
\sphinxstyleliteralstrong{\sphinxupquote{grid}} \textendash{} a grid instance

\item {} 
\sphinxAtStartPar
\sphinxstyleliteralstrong{\sphinxupquote{model}} \textendash{} a 2D model instance

\end{itemize}

\item[{Returns}] \leavevmode
\sphinxAtStartPar
None

\end{description}\end{quote}

\end{fulllineitems}

\index{run\_3d\_model() (in module octapy.tracking)@\spxentry{run\_3d\_model()}\spxextra{in module octapy.tracking}}

\begin{fulllineitems}
\phantomsection\label{\detokenize{octapy:octapy.tracking.run_3d_model}}\pysiglinewithargsret{\sphinxcode{\sphinxupquote{octapy.tracking.}}\sphinxbfcode{\sphinxupquote{run\_3d\_model}}}{\emph{\DUrole{n}{model}}, \emph{\DUrole{n}{grid}}}{}
\sphinxAtStartPar
generate the trajectories for a 3D Model instance
\begin{quote}\begin{description}
\item[{Parameters}] \leavevmode\begin{itemize}
\item {} 
\sphinxAtStartPar
\sphinxstyleliteralstrong{\sphinxupquote{grid}} \textendash{} a grid instance

\item {} 
\sphinxAtStartPar
\sphinxstyleliteralstrong{\sphinxupquote{model}} \textendash{} a 3D model instance

\end{itemize}

\item[{Returns}] \leavevmode
\sphinxAtStartPar
None

\end{description}\end{quote}

\end{fulllineitems}



\chapter{Module contents}
\label{\detokenize{octapy:module-octapy}}\label{\detokenize{octapy:module-contents}}\index{module@\spxentry{module}!octapy@\spxentry{octapy}}\index{octapy@\spxentry{octapy}!module@\spxentry{module}}

\renewcommand{\indexname}{Python Module Index}
\begin{sphinxtheindex}
\let\bigletter\sphinxstyleindexlettergroup
\bigletter{o}
\item\relax\sphinxstyleindexentry{octapy}\sphinxstyleindexpageref{octapy:\detokenize{module-octapy}}
\item\relax\sphinxstyleindexentry{octapy.data}\sphinxstyleindexpageref{octapy:\detokenize{module-octapy.data}}
\item\relax\sphinxstyleindexentry{octapy.get\_data\_at\_index}\sphinxstyleindexpageref{octapy:\detokenize{module-octapy.get_data_at_index}}
\item\relax\sphinxstyleindexentry{octapy.interp\_idw}\sphinxstyleindexpageref{octapy:\detokenize{module-octapy.interp_idw}}
\item\relax\sphinxstyleindexentry{octapy.tools}\sphinxstyleindexpageref{octapy:\detokenize{module-octapy.tools}}
\item\relax\sphinxstyleindexentry{octapy.tracking}\sphinxstyleindexpageref{octapy:\detokenize{module-octapy.tracking}}
\end{sphinxtheindex}

\renewcommand{\indexname}{Index}
\printindex
\end{document}